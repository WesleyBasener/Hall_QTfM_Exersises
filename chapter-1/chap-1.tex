\documentclass[10pt]{article}

\usepackage[margin=1in]{geometry} 
\usepackage{amsmath,amsthm,amssymb, graphicx, multicol, array}

\newcommand{\N}{\mathbb{N}}
\newcommand{\Z}{\mathbb{Z}}

\newenvironment{problem}[2][Problem]{\begin{trivlist}
		\item[\hskip \labelsep {\bfseries #1}\hskip \labelsep {\bfseries #2.}]}{\end{trivlist}}

\begin{document}
	
	\title{Exercises from Chapter 1}
	\author{Wesley Basener}
	\maketitle
	
	\begin{problem}{1.1}
		Beginning with the formula for the sum of a geometric series, use
		differentiation to obtain the identity.\\
		\begin{align*}
			\sum_{n=0}^{\infty}ne^{-An} = \frac{e^{-A}}{(1-e^{-A})^{2}} 
		\end{align*}
	\end{problem}
	
	
	
	\begin{proof}[Solution]
		First, we integrate the left side to get the expression into geometric series form. 
		\begin{align*}
			\int \sum_{n=0}^{\infty} ne^{-An} dA =  \sum_{n=0}^{\infty} -e^{-An}
		\end{align*}
		
		Next, recall that the sum of a geometric series is $\sum_{n=0}^{\infty} ar^{n} = \frac{a}{1-r}$. Using this fact, the previous result can be rewritten,
		\begin{align*}
			\frac{-1}{1-e^{A}}
		\end{align*}
		Finally, we take the derivative with respect to A, to undo our previous integration.
		\begin{align*}
			\frac{d}{dA} \frac{-1}{1-e^{A}} =\frac{e^{-A}}{(1-e^{-A})^{2}}
			\end{align*}
	\end{proof}
	
	\begin{problem}{1.2}
		In Planck’s model of blackbody radiation, the energy in a given frequency $\omega$ of electromagnetic radiation is distributed randomly over all numbers of the form $n \hbar \omega$, where $n = 0, 1, 2, . . . $. Specifically, the likelihood of finding energy $n \hbar \omega$ is postulated to be\\
		\begin{align*}
			p(E=n \hbar \omega) = \frac{1}{Z}e^{-\beta n \hbar \omega},\\
			Z = \frac{1}{1-e^{-\beta n \hbar \omega}}
		\end{align*}
		Where $Z$ is a normalization constant, which is chosen so that the sum over $n$ of the probabilities is 1. Here $\beta = \frac{1}{k_{B}T}$, where T is the temperature and $k_{B}$ is Boltzman's constant. The expected value of the energy, denoted $<E>$, is defined to be
		\begin{align*}
			<E> = \frac{1}{Z} \sum_{n=0}^{\infty} (n \hbar \omega) e^{-\beta n \hbar \omega}.
		\end{align*}
		
		(a) using exercise 1, show that
		\begin{align*}
			<E> = \frac{\hbar \omega}{e^{\beta \hbar \omega} - 1}.
		\end{align*}
		
		(b) Show that $<E>$ behaves like $\frac{1}{\beta} = k_B T$ for small $\omega$, but that
		$<E>$ decays exponentially as $\omega$ tends to infinity.
	\end{problem}
	
	\begin{proof}
		(a) From exercise 1, we can rewrite the sum term as
		\begin{align*}
			\frac{\hbar \omega e^{-\hbar \omega \beta}}{(1-e^{-\hbar \omega \beta})^{2}}
		\end{align*}
		Multiply this by $\frac{1}{Z} = 1-e^{-\hbar \omega \beta}$.
		\begin{align*}
			1-e^{-\hbar \omega \beta}\frac{\hbar \omega e^{-\hbar \omega \beta}}{(1-e^{-\hbar \omega \beta})^{2}}\\
			=\frac{\hbar \omega e^{-\hbar \omega \beta}}{1-e^{-\hbar \omega \beta}}\\
			=\frac{\hbar \omega}{e^{\hbar \omega \beta}-1}
		\end{align*}
		
		(b) Using the Taylor series expansion of $e^{x}$, we can rewrite $<E>$
		\begin{align*}
			\frac{\hbar \omega}{e^{\hbar \omega \beta}-1}\\
			= \frac{\hbar \omega}{(1 + \hbar \omega \beta + \sum_{n=2}^{\infty} \frac{(\hbar \omega \beta)^{n}}{n}) -1}\\
			=\frac{1}{\beta + \sum_{n=2}^{\infty} \frac{(\hbar \omega)^{n-1} \beta^{n}}{n}}
		\end{align*}
		as $\omega$ approaches $0$, the sum disappears and the fraction approaches $\frac{1}{\beta}$.\\
		It is easy to find the limit of $<E>$ as $\omega$ approaches $\infty$ using L'Hospital's rule
		\begin{align*}
			\frac{\hbar \omega}{e^{\hbar \omega \beta}-1} \overset{\mathrm{H}}{=}
			\frac{\hbar}{\hbar \beta e^{\hbar \omega \beta}}
		\end{align*}
	\end{proof}

	
\end{document}